\documentclass[french,12pt]{article}
\usepackage[french]{babel}
\usepackage[T1]{fontenc}
\usepackage[utf8]{inputenc}
\usepackage{amsmath, amssymb}
\usepackage{lmodern}
\usepackage{tabularx}
\usepackage{amsmath}
\usepackage{graphicx}
\usepackage{here}
\usepackage[top = 2cm, bottom = 2cm, right = 2cm, left = 2cm]{geometry}
\usepackage{cite}
\usepackage[final]{hyperref}
\usepackage{listings}
\hypersetup{
	colorlinks=true,
	linkcolor=blue,
	citecolor=blue,
	filecolor=magenta,
	urlcolor=blue         
}

\begin{document}

\title{Critère d'évaluation\\Rendu de TP\\Examen}
\date{12/02/2020}
\maketitle

\section{Evaluation}
L'évaluation des étudiants sera effectué suivant trois critères pondérés de la manière suivante :

\begin{itemize}
	\item Un contrôle continu (coefficient 2)
	\item Un travail collaboratif (coefficient1)
	\item Une évaluation des connaissances (coefficient1)
\end{itemize}

Le contrôle continu prendra la forme d'un rendu de travaux pratiques.\\
L'examen final sera divisé en deux parties. Les 30 premières minutes seront dédiées à l'étude d'une problématique issue d'un cas réel en groupe de deux à cinq personnes. La fin de l'examen (1h30) évaluera les questions de cours.

\section{Rendu de TP}
Le compte rendu du TP son code source devra être rendu au plus tard la veille de l'examen final (16/03/2020) au soir. Il devra être rédigé en français ou en anglais. Etant donné que dans les années futures vous allez surement être amené à rédiger des documents de synthèse, je vous invite à suivre ce plan.

\begin{itemize}
	\item Introduction (Présentation du contexte)
	\item Table des symboles (Index des symboles utilisées)
	\item Table des figures (Index des images)
	\item Production (Description des travaux)
	\item Résulats (Données brute obtenues : La génération d'une image prend 10 secondes pour une résolution de 1024x1024. Si la résolution est de 16 par 16 on obtient un crash, ...)
	\item Discussion des résultats (Critique des résultats. Par exemple : On s'attendait a tel résultat et on a obtenu ceci)
	\item Conclusions (Rédaction du bilan, des pistes d'amélioration/optimisation, ouverture sur d'autre sujet, ...)
\end{itemize}

\subsection{Table des symboles}
Il est important dans un rapport de conserver une cohérence entre les symboles que vous utilisez. Il ne faut pas que vous choisissiez $i$ pour désigner un indice, puis que vous utilisiez $i$ dans une formule utilisant des nombres complexe. Pour éviter ceci on utilisera la table des symboles qui regroupera toutes les notations que vous utiliserez avec leurs définitions. Par exemple :

\begin{itemize}
	\item $t$ : Le nombre d'itérations.
	\item $\mathbf{C}$ : Une couleur.
	\item $M$ : l'ensemble de Mandelbrot.
\end{itemize}

On notera que les scalaires sont en minuscule, les vecteurs en majuscule et en gras.

\subsection{Rédaction d'une équation}
On rédigera une équation de la manière suivante : 

\begin{equation} \label{eq1}
\begin{split}
S & = (x-x_0)^2+(y-y_0)^2+(z-z_0)^2 = r^2, \\
 & = (x-x_0)^2+(y-y_0)^2+(z-z_0)^2 -r^2 = 0;
\end{split}
\end{equation}

On notera que l'équation est indexée. On mettra des virgules pour chaque ligne de la démonstration et un point virgule pour la fin d'une démonstration.

\section{Table des Figures}
Encore une fois on listera les images avec leurs descriptions. On utilisera la même description pour commenter l'image.
\begin{itemize}
	\item Figure 1 : Ensemble de Mandelbrot
\end{itemize}

\begin{figure}[H]
	\centering
	\includegraphics[scale=0.1]{figures/Mandelbrot.png}
	\caption{Ensemble de Mandelbrot}
\end{figure}

\section{Examen}
\subsection{Travail en groupe}
Pour la première partie de l'examen vous travaillerez en groupe sur un sujet comme par exemple comment faire évoluer un système météorologique sur un GPU.
\subsection{Evaluation des connaissances}
Cette deuxième partie sera constituée simplement de question de cours vu durant le premier cours et durant les TPs.


\end{document}